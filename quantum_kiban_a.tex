%\documentclass[11pt,a4paper,uplatex,dvipdfmx]{ujarticle} 		% for uplatex
\documentclass[11pt,a4j,dvipdfmx]{jarticle} 					% for platex
\input{pieces/form00_header} % pieces
\input{pieces/kakenhi7} % pieces
\input{pieces/form01_header} % pieces
\input{pieces/form02_2024_header} % pieces
\input{pieces/hook3} % pieces
%#Name: kiban_a
\input{pieces/form04_jsps_headers} % pieces
\input{pieces/form04_kiban_a_header} % pieces
% ===== Global definitions for the Kakenhi form ======================
% 基本情報
%
%------ 研究課題名  -------------------------------------------
\newcommand{\研究課題名}{誤り耐性量子コンピュータに向けた誤り訂正技術の開発(仮)}

%----- 研究機関名と研究代表者の氏名-----------------------
\newcommand{\研究機関名}{東京大学}
\newcommand{\研究代表者氏名}{寺師弘二}
\newcommand{\me}{\underline{\underline{K.~Terashi}}} 
%---- 研究期間の最終年度 ----------------
\newcommand{\研究期間の最終元号年度}{10}  %令和で、半角数字のみ
%========================================

\input{pieces/inst_general_images} % pieces

% user07_header
% ===== my favorite packages ====================================
% ここに、自分の使いたいパッケージを宣言して下さい。
\usepackage{wrapfig}
\usepackage{amsbsy}
%\usepackage{mb}
%\DeclareGraphicsRule{.tif}{png}{.png}{`convert #1 `dirname #1`/`basename #1 .tif`.png}
\usepackage{lineno}
\usepackage{comment}
\usepackage{amsfonts}

% ===== my personal definitions ==================================
% ここに、自分のよく使う記号などを定義して下さい。
\newcommand{\mybf}[1]{{\bfseries\sffamily#1}}

\newcommand{\maruone}{\raise0.2mm\hbox{\textcircled{\scriptsize{1}}}}
\newcommand{\marutwo}{\raise0.2mm\hbox{\textcircled{\scriptsize{2}}}}
\newcommand{\maruthree}{\raise0.2mm\hbox{\textcircled{\scriptsize{3}}}}
\newcommand{\marufour}{\raise0.2mm\hbox{\textcircled{\scriptsize{4}}}}
\newcommand{\marufive}{\raise0.2mm\hbox{\textcircled{\scriptsize{5}}}}

% ----- 業績リスト用 -------------
\newcommand{\paper}[6]{%
	% paper{title}{authors}{journal}{vol}{pages}{year}
	\item ``#1'', #2, #3 {\bf #4}, #5 (#6).			% お好みに合わせて変えてください。
}

\newcommand{\etal}{\textit{et al.\ }}
\newcommand{\ca}[1]{*#1}	% corresponding author;   \ca{\yukawa}  みたいにして使う
\newcommand{\invitedtalk}{招待講演}

\newcommand{\iiyama}{\underline{Y.~Iiyama}}	
\newcommand{\chen}{\underline{S.~Chen}}

\newcommand{\prl}{Phys.\ Rev.\ Lett.\ }		% よく使う雑誌も定義すると楽

% ===== 欄外メモ ==================
\newcommand{\memo}[1]{\marginpar{#1}}
%\renewcommand{\memo}[1]{}	% 全てのメモを表示させないようにするには、行頭の"%"を消す

\input{pieces/hook5} % pieces

\begin{document}
\input{pieces/hook7} % pieces
%#Split: 01_purpose_plan  
%#PieceName: p01_purpose_plan
\input{pieces/p01_purpose_plan_00}
\section{1 研究目的、研究方法など}
%    <<最大 6ページ>>

%s02_purpose_plan_with_abstract
%\JSPSInstructions		% <-- 留意事項。これは消すか、コメントアウトしてください。
\noindent
\textbf{(概要)}\\
%begin 研究目的及び研究計画の概要空行付き ====================
現在の量子コンピュータは、ハードウェア由来のさまざまなノイズによるエラーを訂正できない過渡期の計算機である。量子コンピュータを実用的な問題に活用するためには、計算途中に随時エラーを検知し訂正を行う、\mybf{誤り耐性量子コンピュータ(FTQC)}が必要である。FTQCでは、論理量子ビットを構成する物理量子ビットを測定してエラーを検知し、その結果から物理量子ビットのエラーを訂正するユニタリー演算を決定し、訂正を行う。この\mybf{エラー検知のための信号処理とエラーから回復するための操作の決定に多くの計算資源が必要}であり、 FTQC実現の障害の一つになっている。本研究は、\mybf{\ul{量子状態の読み出しとエラー検知・エラー訂正の基本操作に焦点を当て、その基礎技術をハードウェアとソフトウェアの両面で確立}}する。特にジョセフソン接合を用いた小型高周波素子の開発と読み出し信号の多重化、問題が持つ物理的な対称性に基づいたエラー訂正手法の開発によって、\mybf{\ul{実用的な科学問題への量子コンピュータの応用}}を加速し、将来的に\mybf{\ul{スケールアップ可能なエラー訂正の基礎技術を実現}}する。
%end 研究目的及び研究計画の概要空行付き ====================

\noindent
\rule{\linewidth}{1pt}\\
\noindent
\textbf{(本文)}
%begin 研究目的と研究計画	====================

\noindent{\bf (1)本研究の学術的背景、研究課題の核心をなす学術的「問い」}\\
量子コンピュータ、特にノイズによるエラーを訂正できない現在の量子コンピュータ(NISQと総称される)は、Googleの超伝導量子コンピュータによる量子超越性の実証など、特定の限られた問題では古典計算を超える性能を発揮できることが示されている。実用的な問題に対しては、NISQコンピュータの優位性を実証できる段階にはまだ至っていないが、\mybf{スピン多体系のシミュレーションなど、物理的に意味のある問題へのNISQの応用が進んでおり、問題によっては古典計算と同等かそれ以上の結果が得られつつある}~[\ref{IBM_Utility}]。
NISQを有効に活用するためには、量子計算を多数回繰り返してノイズの性質をモデル化し、その結果を使って期待値からエラーによるバイアスを取り除く「エラー緩和」の技術が必要である。しかし、エラー緩和を使ってバイアスのない期待値を推定するためには、量子回路中で起こるエラーの発生率に対して指数的な数のサンプリングを行う必要がある。そのため、通常多数の量子ビットを必要とする実用的な問題に対しては、エラー緩和を使って優位性を示すことは一般的に難しいと考えられている。

将来的に、幅広く量子コンピュータを実用問題に活用するためには、\mybf{計算途中で随時エラーを検知し訂正を行う、誤り耐性量子コンピュータ(FTQC)が必要}になる。FTQCでは、多数の物理量子ビットを組み合わせて論理量子ビットを構成する。そして、物理量子ビットに接続した補助量子ビットを新たに用意し、その測定から物理量子ビットのエラーを検知し訂正を行う。FTQCでのエラー訂正は、一般的に\mybf{多数の物理量子ビットで起こるエラーを検知するための信号処理}と、\mybf{エラーからリカバリーするためのユニタリー演算を決定する古典計算(デコーダ)}に多くの計算資源が必要であり、FTQC実現の障害の一つになっている。言い換えれば、\mybf{\ul{エラー検知のための量子状態の操作と読み出し、デコーダによるリカバリー演算の決定を繰り返し行うというエラー訂正の基本操作を、その基礎技術から確立する必要がある}}ことを強く示唆している。

超伝導量子コンピュータ開発を先導する企業の一つであるIBMは、2025年までに4,158量子ビット、その後2033年までに10万量子ビットを実現すべく研究を進めている。数千物理量子ビットのシステムはエラー訂正の初期実装に十分であり、数万$\sim$数十万量子ビットではFTQCが現実味を帯びてくる。本研究課題は、FTQC時代の到来を見据え、\mybf{\ul{『拡張性(スケーラビリティ)を持つエラー訂正の基礎技術を確立できるか?』}}を問うものである。\\


\noindent{\bf (2)本研究の目的及び学術的独自性と創造性}\\
本研究の大きな目的は、科学や社会応用など実用的な問題に対して、古典計算機より優れた性能を持つ量子コンピュータを実現することである。そのために、誤り耐性量子コンピュータに必要なエラー訂正について、その基礎技術をハードウェアとソフトウェアの両面で開発する。


超伝導量子コンピュータでの論理量子ビットのエラー訂正は、補助量子ビットの逐次測定と訂正用マイクロ波パルスのフィードバックによって行われる。
ゲート演算の度に補助量子ビットの状態は測定され、古典計算機がその読み出し情報からパリティ計算を行ってエラーの有無を判定する(デコード)。
エラーを検出した場合、対応する物理量子ビットにエラーを訂正するマイクロ波パルスを照射して状態を回復する。
現在の超伝導量子コンピュータは、量子ビットの読み出しに約1~$\mu$s、古典計算機におけるデコードには高速のFPGAを用いてもそれぞれ数~$\mu$sの時間を消費する。
演算に使用する制御ゲート時間が20~ns--400~nsであることを考えると、この\mybf{エラー訂正のシークエンスが実行時間の支配的な成分}となる。
エラー訂正の実行中に量子ビットがコヒーレンスを失い訂正が失敗するリスクも上がるため、 \mybf{\ul{エラー訂正時間の短縮はFTQC時代においては喫緊の課題}}である。
また、エラー訂正では大量の補助量子ビットが必要となるため\footnote{現在の単一量子ビットのコヒーレント時間100~$\mu$s--1~msでは一つの論理量子ビットに対しておよそ100個の補助量子ビットが必要となる。将来量子ビットのコヒーレント時間が伸びるに従って改善される余地はあるが、ここ5年のうちに桁が変わるような変化が起こるとは考えづらい。}、\mybf{量子ビット数に対してスケールするハードウェア実装モデルの確立}も重要である。
特に量子ビットの読み出し・制御に用いる同軸ケーブル、読み出し信号の増幅に使う低温アンプ、ノイズから量子ビットを保護するアイソレータ・サーキュレータなどの高周波素子の数は実装する量子ビット数に比例して増えていくが、これらは数立方センチメートルの物理的なスペースを取るため、1,000量子ビットを超える実装では典型的な希釈冷凍機のボリュームに収めるのは極めて困難である。
そのため\mybf{単一配線で複数の量子ビットの読み出しを行う\ul{信号多重化}}と、これら\mybf{\ul{高周波素子の小型化}}は\mybf{\ul{スケールアップを考える上で避けて通れない課題}}である。
本研究ではこれらのハードウェアの課題に対し、\mybf{(1)ジョセフソン接合を用いた小型高周波素子・エラーデコーダの製作}、\mybf{(2)時間領域における信号多重化}の2つの方向から取り組む。

さまざまな問題に幅広く活用できる量子コンピュータを実現するには、ハードウェアからアルゴリズム、ソフトウェアの各階層で多様なユースケースに対応する必要があり、開発に多くの時間がかかる。解きたい問題に適した設計にすることで応用を加速する\mybf{「Co-design」}の考え方は過渡期の今は特に有効であり、その考え方を誤り訂正技術の開発に取り入れる。そこで、本研究のソフトウェア面での開発では、\mybf{\ul{量子コンピュータの優位性を示すことができる可能性がある実用的な問題として基礎物理や物性物理での量子多体系シミュレーションを考える}}。これらの\mybf{物理系が持つ対称性を新たにエラー訂正に導入する}ことで、よりスケーラブルなエラー訂正計算の手法を作る。対称性を用いるエラー緩和の手法はすでに提案されているが、その考えをエラー訂正に導入することは新しい試みであり、エラー緩和の手法をベースにすることで実現可能性も高い。

\mybf{対称性は物理系が持つ普遍的な性質であり、\ul{対称性を用いたエラー訂正手法はどのようなハードウェア環境でも原理的に応用が可能}}であり、高い創造性を持つ。将来の量子コンピュータのアーキテクチャーとして、複数の量子チップを相互接続して計算を行う\mybf{分散型量子計算}が検討されている。このアーキテクチャーでは、解きたい問題を小さな問題に分割し、個々の量子チップで分割した問題を解いた後に結果を統合することが必要になる。問題の対象になる量子状態(例えばスピン系の基底状態)が対称性を持っている場合、その状態はヒルベルト空間全体ではなく一部の部分空間のみを張るため、対称性は問題を分割する一つの指針を与える。\mybf{\ul{本研究が進める対称性を用いたエラー訂正手法は、将来的にFTQCが分散型量子計算へと発展していく場合にも応用が可能}}である。\\


\noindent{\bf (3)着想に至った経緯、国内外の研究動向と本研究の位置づけ}\\
IBMは現有量子チップのエラー緩和技術を高度化し、それによってNISQコンピュータを実用的な問題に応用する方向で開発を進めているが、論理量子ビットをどう構成してエラー訂正を行うかについては、まだ研究の初期段階であり明確な方針はない。Googleは物理量子ビットの数を増やした時に論理量子ビットのエラー率が下がることを初めて実証する~[\ref{Google}]など、エラー訂正技術の開発を中心に進めている。しかし、量子ビットの数を増やして単一の量子チップを大型化する方向に進んでおり、将来的にスケーラビリティをどう確保していくのか不明な点が多い。

エラー訂正符号の一つである\mybf{表面符号}は、物理量子ビットに対するエラーレートの要求が1\%程度と比較的低いことと、平面上に配置した量子ビットにも使えることから最も研究が進んでいる。しかし、エラーに対してより強い耐性を持たせるために符号距離$d$(補正できるエラーの数の目安)を大きくすると、1物理量子ビットあたりの論理量子ビットの作成レート$R$は$1/d^2$に比例して小さくなるため効率が悪く、初期のFTQCには適さない。作成レート$R$が$1/d$に比例する訂正符号の一つとして\mybf{低密度パリティ検査(LDPC)符号}が知られており、古典計算でのエラー訂正手法として広く使われてきた。その手法を取り入れた量子LDPC符号の研究が最近進展してきたが~[\ref{LDPC2}]、表面符号に比べると研究の初期段階にあり理解が不十分である。特に、量子状態の生成・変換・測定などさまざまな過程で起こる現実的なノイズに対しては、量子LDPC符号の効率的なデコーダが知られておらず、ハードウェアへの実装はまだ実現していない。

NISQからFTQCへの過渡期において、量子コンピュータの優位性を示すことのできるアプリケーションに適しているのは基礎科学や物性物理の問題である。これらの物理の問題は、多くの場合対称性を内在している。例えば素粒子物理を記述するゲージ理論のシミュレーションでは、粒子数保存などの対称性を表現する量子ゲートを使うことで、基底状態のシミュレーションが効率的に行えることが分かっている。そのような\mybf{対称性は誤り訂正にも応用可能であり、その手法によって\ul{現実的な物理問題での量子コンピュータの優位性を早期に示すことができる}のではないか}と考え、本研究の着想に至った。また、以下のように本研究グループは\mybf{研究に必要なハードウェアとソフトウェアの技術}をすでに持っており、そのことも本研究を現実的なものにしている。

\underline{\bf ハードウェア}\vspace{-2mm}
\begin{itemize}
\item ジョセフソン接合の製作に必要な微細加工の技術に精通し、それを用いた\mybf{超伝導量子素子の製作に成功}している。\vspace{-2mm}
\item \mybf{希釈冷凍機に実装する環境を構築}し、コヒーレンス時間の測定など\mybf{量子ビットとしての性能評価を行う}ことができる。\vspace{-2mm}
\item 長年取り組んでいる大型素粒子実験を通じ、\mybf{大量の読み出しチャンネルの時間制御技術を有している}。
\end{itemize}

\underline{\bf ソフトウェア}\vspace{-2mm}
\begin{itemize}
\item 素粒子物理の基本原理に精通しており、\mybf{対称性を量子回路に実装する技術}を持っている。\vspace{-2mm}
\item エラー検知には物理量子ビットと補助量子ビット間で大量の2量子ビットゲートを使うが、その超伝導量子コンピュータでの実装に使われる\mybf{「交差共鳴パルス」のシミュレーションや最適なシークエンス設計に精通}している。
\end{itemize}

本研究が目指すのは、\mybf{誤り訂正を現実的に実現するための基礎技術を確立}することである。表面符号や量子LDPC符号などさまざまな誤り訂正符号が提案され、今後発想の異なる全く新しい訂正符号が提案される可能性もある。しかし、本研究はその\mybf{\ul{超伝導量子コンピュータ実装に必要なハードウェア技術を基礎から作る}ものであり、重要性が失われることはない}。本研究で取り組む\mybf{対称性を用いたエラー訂正手法は物理系が持つ普遍的な性質を用いるため、その基本原理はハードウェア環境に依存しない}。超伝導電子回路や半導体、光、イオンなど多様な技術があり、どのような技術によってFTQCを実現できるのかまだ明確ではないが、\mybf{本研究が考案する\ul{エラー訂正手法の原理はさまざまなハードウェアに広く応用する}}ことができる。\\


\noindent{\bf (4)何をどのように、どこまで明らかにしようとするのか}\\
\noindent \mybf{\ul{\maruone\ 物理問題の設定}}

%\begin{comment}
\begin{wrapfigure}{r}{10.5cm}
	\begin{center}
		\vspace{-1.4cm}
		%\includegraphics[width=10.5cm]{figs/trotter}\vspace{-0.4cm}
		\includegraphics[width=10.5cm]{figs/trotter.png}\vspace{-0.4cm}
		\caption{\small{(a)ハミルトニアン時間発展$|\psi(t)\rangle=e^{-iHt}|\psi(0)\rangle$の鈴木-トロッター分解を用いた量子回路。(b)$n$量子ビット系で粒子数$m$を保存する量子回路(上)は2量子ビットゲート(下)で実装可能。(c)誤差0.1\%でシミュレートするために必要な量子ビット数-制御NOTゲート数の関係を、鈴木-トロッター近似のオーダーと時間ステップを変えて評価する。}
		\label{fig:z2}}\vspace{-0.7cm}
	\end{center}
\end{wrapfigure}
%\end{comment}

初期のFTQCでの優位性を示すことのできる問題として、素粒子物理を記述する\mybf{ゲージ理論の量子シミュレーション}を考える。特に\mybf{2次元の格子ゲージ理論($\pmb{U(1)}$ゲージ場を離散化した$\pmb{\mathbb{Z}_2}$ゲージ理論)}では、外部電場が存在する時に\mybf{物質が局所的な空間に閉じ込められる現象}が起こる。この現象を古典的にシミュレートするには、時間発展の各時間ステップごとに外部電荷の設定を変えて繰り返し計算する必要があるため、高い計算コストがかかる。量子シミュレーションは時間発展をより効率的に行えるため、量子コンピュータの優位性を示すことができる可能性がある。
本研究では、まず2次元$\mathbb{Z}_2$ゲージ場のハミルトニアンによる時間発展を、鈴木-トロッター分解(図~\ref{fig:z2}(a))を使ってユニタリー演算子の積に近似する。次に、
量子状態が保持する対称性(空間反転パリティ、粒子数)を取り入れた量子ゲートと一般的な量子ゲートを使って量子回路に実装する(図~\ref{fig:z2}(b))。
シミュレータを使って時間発展シミュレーションを行い、時間発展状態のエネルギーや外部電場による閉じ込め効果を測定する。\vspace{-4mm}\\

\underline{\bf マイルストーン}\vspace{-2mm}
\begin{itemize}
\item 20量子ビットの系に対して、特定の鈴木-トロッター近似のオーダー$\Omega$と時間ステップ$N$に対して、誤差1\%以下でのエネルギー測定を行う。\vspace{-2mm}
\item $\Omega$と$N$を変え、誤差0.1\%以下でのシミュレーションに必要な2量子ビットゲート数を評価する。その評価結果を20量子ビット以上の系に対して外挿する(図~\ref{fig:z2}(c))。\vspace{-2mm}
\item 対称性を取り入れた量子ゲートを使うことで、一般的な量子ゲートを使った場合の約半分の2量子ビットゲート数に抑えられることを示す(図~\ref{fig:z2}(c))。
\end{itemize}





\noindent \mybf{\ul{\marutwo\ 対称性を用いたエラー訂正手法の開発}}

量子ビットは状態操作や読み出しのために外部環境と相互作用しなければならないため、外的な要因によってエラーが起こることは避けがたい。ある対称性を持つ物理系を考える場合、その\mybf{保存量を繰り返し測定し途中で結果が変わった場合、測定の間でエラーが起こったと推定}できる。そのための典型的なエラー検知の方法はスタビライザー測定と呼ばれており、物理量子ビットでビット反転や位相反転が起こった時に、測定した補助量子ビットのパリティ検査の結果からエラーを検知する。
本研究では、2次元$\mathbb{Z}_2$ゲージ場での保存量として粒子数や空間反転パリティに着目し、それらの量を測定する対称オペレータを定義する。
閉じ込め効果など興味のある観測量の測定をしながら、対称性を保っているか(エラーが起こっていないか)判定するためには、\mybf{観測量と可換な対称オペレータを同時に測定}する必要がある。エラー緩和の先行研究をベースに、この同時測定のための手法を量子回路として実行する。
対称オペレータの測定結果が初期状態での予想量と一致していない場合に、対称性を回復させる(エラーを訂正する)ためのデコーダを開発する。\vspace{-4mm}\\

\underline{\bf マイルストーン}\vspace{-2mm}
\begin{itemize}
\item 20量子ビットの系に対して、粒子数を対称オペレータ、時間発展状態のエネルギーを物理量として測定する量子回路を実装する。そこに10\%のエラーレートを持つ一般的な分極ノイズモデルを適用し、対称性によるエラー検知が可能であることを示す。\vspace{-2mm}
\item 初期状態の粒子数に誤差1\%で訂正するためのユニタリー演算子を機械学習モデルを使って決定する(デコーダの開発)。\vspace{-2mm}
\item 量子コンピュータの実機を使って検証実験を行い、エネルギー測定の精度がエラー訂正によって向上することを確認する。
\end{itemize}



\noindent \mybf{\ul{\maruthree\ スケーラブルな超伝導量子コンピュータのためのハードウェア開発}}

\begin{wrapfigure}{r}{10.8cm}
	\begin{center}
		\vspace{-1cm}
		%\includegraphics[width=10.5cm]{figs/hard}\vspace{-0.4cm}
		\includegraphics[width=10.8cm]{figs/hard.png}\vspace{-0.4cm}
		\caption{\small{(a)ジョセフソン接合ベースの整流素子([\ref{JJ_reciprocle}]から引用)。\maruone 一方向からの光だけを通しやすい性質を持つ。\marutwo 実際のデバイスの写真。\maruthree 3個のジョセフソン接合をもつ超伝導磁束量子ビット(中央の点線で囲まれた領域)と、左右に伸びる二つの共振器から構成される。\marufour 中央点線部分の拡大図。左の白四角に1個、右の白四角に2個のジョセフソン接合が作られている。(b)周波数分割の多重化(上)と時間分割の多重化(下)のイメージ。}
		\label{fig:hard}}\vspace{-0.7cm}
	\end{center}
\end{wrapfigure}


アイソレータやサーキュレータは高周波マイクロ波の整流を行う素子であり、超伝導量子コンピュータでは読み出しケーブルから逆流してきたノイズがアンプを通じて増幅されるのを阻止する重要な役割を果たす。
通常これらは偏極させた強磁性体が偏極方向に対して特定の方向にしか光を通さない性質(相反性)を利用しているが、偏極させるための永久磁石の磁場を閉じ込めるためのシールドが必要であり、これが小型化のボトルネックとなっている。ところが近年、強磁性体の代わりに\mybf{ジョセフソン接合の非線形光学効果を利用して相反性を実現}したという報告がなされ~[\ref{JJ_reciprocle}]、このボトルネックが解消に向かう可能性が示唆されている。特に\mybf{微細加工で製作するため回路が小型化でき}(図~\ref{fig:hard}(a))、\mybf{超伝導量子ビットと同じチップに搭載できれば配線の削減も可能}となるため期待は大きいが、帯域の狭さや相反性の大きさ、挿入損失の大きさなど課題は多い。これらの課題を改善し、従来型素子の性能に近づけるのが本研究の目的である。
%ジョセフソン接合を応用した磁束量子ビットを用いたデコーダが実際に提案されている

超伝導量子ビットの読み出しは、量子ビットに照射したマイクロ波パルスの透過成分の位相測定を通じて行われる(分散読み出し)。一般的には周波数領域でのマイクロ波の多重化が使われているが、限られた周波数帯域の中では周波数が近づくために信号の混線を起こしやすい。本研究では\mybf{時間領域での多重化技術}(図~\ref{fig:z2}(b))\mybf{を開発し、システム全体の大型化}の研究を進める。\vspace{-4mm}\\

\underline{\bf マイルストーン}\vspace{-2mm}
\begin{itemize}
%\item ジョセフソン接合ベースの読み出し用量子素子によって、通常用いられる分散読み出しに対して3倍の高速化を実現する。\vspace{-2mm}
\item ジョセフソン接合ベースの超小型サーキュレータ/アイソレータを開発し、ノイズ抑制のための要求性能を満たしつつ希釈冷凍機内での占有体積を10分の1以下にする。\vspace{-2mm}
\item 周波数の多重化で一度に読み出せる量子ビット数を50\%上回る数の量子ビットを、同じ時間内に時間の多重化によって読み出す。
\end{itemize}


\noindent{\bf (5)目的を達成するための準備状況}\\
最初のステップである物理問題の設定については、本研究に先行して、1次元$U(1)$ゲージ理論である\mybf{シュウィンガー模型}のシミュレーションに取り組んできた。シュウィンガー模型はシンプルでありながら、カイラル対称性(粒子のスピンの向きに関する変換対称性)の破れなど、現実世界の非可換ゲージ理論と同じ性質を共有しており、場の理論のシミュレーションではベンチマークとして良く活用される。本研究グループは、\mybf{シュウィンガー模型でのエネルギー基底状態やその時間発展状態を近似する状態を、パラメータ化した量子回路を使って量子コンピュータに実装する}研究を進めてきた。基底状態については変分量子固有値ソルバー、時間発展状態については時間依存変分量子シミュレーションと呼ばれる手法を使ってシミュレーションを行っている。\mybf{1次元$\pmb{\mathbb{Z}_2}$ゲージ理論のハミルトニアン時間発展を鈴木-トロッター分解の方法を使ってシミュレーションを行い、その状態が\ul{電場の有無によって閉じ込め相と非閉じ込め相に別れていくことなどを確認}}した(図~\ref{fig:status}(a))~[\ref{Terashi_QML_Qdata}]。
物理系が持つ対称性を考慮した量子回路の設計は、本研究の重要なステップである。本研究グループはシュウィンガー模型の時間発展シミュレーションの研究の中で、\mybf{粒子数を保存する量子ゲートを実装した回路を使って\ul{長時間の時間発展シミュレーションを行うことに成功}}している。対称性を持つ量子回路が張るヒルベルト空間は全空間の一部に限定されるため、量子回路のパラメータ決定を効率良く行うことができる。この手法を$\mathbb{Z}_2$ゲージ理論のシミュレーションに応用することは比較的容易である。


\begin{wrapfigure}{r}{11.3cm}
	\begin{center}
		\vspace{-1cm}
		%\includegraphics[width=11cm]{figs/result}\vspace{-0.3cm}
		\includegraphics[width=11.3cm]{figs/result.png}\vspace{-0.4cm}
		\caption{\small{(a)1次元$\mathbb{Z}_2$ゲージ理論で、物質場が時間発展と共に拡散していく様子(非閉じ込め, 左)。右は拡散しない場合(閉じ込め)。鈴木-トロッター分解を用いた量子回路で時間発展状態をシミュレートし、閉じ込め相と非閉じ込め相の相分類を行った(下)~[\ref{Terashi_QML_Qdata}]。(b)本研究グループによって製作された超伝導量子ビットの写真。中央の$200~{\rm nm}\times200~{\rm nm}$の領域にジョセフソン接合が作られている。低温科学センターにある希釈冷凍機(右下)と、初期に製作した超伝導量子ビットが$|1\rangle$状態から$|0\rangle$に脱励起するまでの時間($T_1$)の測定結果(左下)。$T_1=7~\mu{\rm s}$をすでに達成。}
		\label{fig:status}}\vspace{-1cm}
	\end{center}
\end{wrapfigure}


ジョセフソン接合ベースのハードウェア開発には、超伝導量子素子の製作に必要な\mybf{微細加工技術}と\mybf{希釈冷凍機を含めた実装・テスト環境}が必要である。量子ビットの微細加工と性能評価をすでに開始しており、\mybf{10~$\boldsymbol{\mu}$s程度のコヒーレント時間を持つ\ul{超伝導トランズモンの製作にも成功}}している(図~\ref{fig:status}(b))。超伝導量子ビットの状態($|0\rangle$と$|1\rangle$)の違いを共振器信号の位相差として読み出す\mybf{分散読み出しにも成功}しており、本研究に必要な基礎技術はすでに獲得している。\vspace{-3mm}\\


\noindent{\bf 研究者の役割}\\
研究協力者であるポスドク研究者・大学院生とともに、研究代表者が研究課題{\maruone}を担当する。課題{\marutwo}については、古典計算ソフトウェアと超伝導量子ビットの物理シミュレーションの知識が豊富である研究分担者(飯山)が主導して進める。課題{\maruthree}については、超伝導量子素子の製作に成功し、マイクロ波のエンジニアリング技術にも精通する研究分担者(陳)が中心となって進める。


\vspace*{0.5zw}
\noindent\textbf{参考文献} \vspace{-2mm}
\begin{enumerate}
	\renewcommand{\labelenumi}{[\arabic{enumi}]}
	\item \label{IBM_Utility} ``Evidence for the utility of quantum computing before fault tolerance'',
		Y.~Kim {\it et al.}, {\it Nature} {\bf 618}, 500 (2023).\vspace{-3mm}
	\item \label{Google} ``Suppressing quantum errors by scaling a surface code logical qubit'', 
		Google Quantum AI, {\it Nature} {\bf 614}, 676 (2023).\vspace{-3mm}
	\item \label{LDPC2} ``Constant-Overhead Quantum Error Correction with Thin Planar Connectivity'', 
		M.~A.~Tremblay {\it et al.}, {\it Phys. Rev. Lett.} {\bf 129}, 050504 (2022).\vspace{-3mm}
	%\item \label{AQCEL} ``Quantum Gate Pattern Recognition and Circuit Optimization for Scientific Applications'',
	% 	W.~Jang, \me {\it et al.}, arXiv:2102.10008 (2021).\vspace{-3mm}
	%\item \label{NTT} ``Cryogenic operation of NanoBridge at 4 K for controlling qubit'', 
	%	K.~Okamoto {\it et al.}, Jpn. J. Appl. Phys. {\bf 61}, SC1049 (2022).\vspace{-3mm}
	%\item \label{Pulse} ``Josephson Microwave Sources Applied to Quantum Information Systems'',
	%	A.~J.~Sirois {\it et al.}, {\it IEEE Transactions on Quantum Engineering}, {\bf 1}, 1 (2020).\vspace{-3mm}
	%\item \label{Passive} ``Two-resonator circuit quantum electrodynamics: A superconducting quantum switch'', 
	%	M.~Mariantoni {\it et al.}, {\it Phys. Rev. B} {\bf 78}, 104508 (2008).\vspace{-3mm}
        \item \label{JJ_reciprocle} ``Microwave quantum diode'',
                R.~Upadhyay {\it et al.}, arXiv:2304.00799.\vspace{-3mm}
		
\end{enumerate}

%end 研究目的と研究計画	====================

\input{pieces/p01_purpose_plan_01}

%#Split: 02_abilities  
%#PieceName: p02_abilities
\input{pieces/p02_abilities_00}
\section{2 応募者の研究遂行能力及び研究環境}
%    <<最大 2ページ>>

% s14_abilities
%\PapersInstructions		% <-- 留意事項。これは消すか、コメントアウトしてください。
%begin 応募者の研究遂行能力及び研究環境 ====================
\noindent{\bf (1)これまでの研究活動}\\
応募者は欧州原子核研究機構(CERN)にある大型ハドロン加速器(LHC)を用いた大型実験ATLASに参画し、高エネルギー素粒子実験を最先端の現場で主導してきた。\mybf{ATLAS検出器のデータ取得システムの構築・運転から物理データ解析、次世代加速器実験への増強計画}まで豊富な知識と経験を有しており、それらの技術をもとに\mybf{量子コンピュータのアルゴリズム開発、量子シミュレーション、量子ソフトウェアとハードウェアの開発}を進めている。\vspace{-2mm}\\

高エネルギー素粒子実験への応用に向け、研究代表者(寺師)は2019年初頭から量子コンピュータの研究を開始した。カナダD-Wave社の量子アニーリングマシンを使い、検出器信号の組み合わせを最適化することでその測定対象の粒子の飛跡を再構成するという研究に、米国ローレンスバークレー国立研究所の研究者とともに取り組んだ。この種の取り組みとしては世界的に最初期のものであり、量子アニーリングによって高い再構成効率が実現できることを示した~[\ref{Saito_CHEP}]。
その後、\mybf{\ul{パラメータ化した量子回路を用いてデータの相関関係を学習する「量子機械学習」の手法を素粒子実験のデータ解析に取り入れ、比較的少数のデータや学習パラメータで古典機械学習手法に匹敵する学習性能を持ちうることを示した}}~[\ref{Terashi_QML_selection}]。量子機械学習で古典データを学習する場合、まずデータを量子状態に符号化し、その状態をパラメータ化した量子回路で変換し測定する。教師あり学習では、その測定結果がデータの教師ラベルに合うようにパラメータを最適化することで学習を行う。量子機械学習は\mybf{量子状態への符号化や学習回路の設計によっては学習が困難になる}ことが知られており(\mybf{勾配消失問題})、その解消に向けた研究を現在進めている。
また本研究課題に先行して、研究課題{\maruone}{\marutwo}に必要な場の理論の量子多体系シミュレーションにも取り組んできた。ポスドク研究者と共に、\mybf{\ul{シンプルな1次元シュウィンガー模型での基底状態}}や\underline{\mybf{$\pmb{\mathbb{Z}_2}$ゲージ理論の時間発展状態のシミュレーション}}に取り組み、\mybf{\ul{生成された状態が期待される性質やエネルギーを持つことを確認}}した。ここで生成したシミュレーションデータを使って、\mybf{IBMチューリッヒの研究者とともに量子状態(波動関数)から物理系の性質を直接学習する量子機械学習の応用}に取り組んだ~[\ref{Terashi_QML_Qdata}]。ここではデータの符号化が必要ではなく、かつ古典畳み込みニューラルネットワーク(CNN)に着想を得た\mybf{量子CNN}を学習回路として使うことで、勾配消失問題に強い学習が可能である。この学習モデルを使い、\mybf{\ul{シュウィンガー模型ではカイラル対称性の破れ}}、\underline{\mybf{$\pmb{\mathbb{Z}_2}$ゲージ理論では外部電場による物質場の閉じ込め}}に由来する\mybf{\ul{相転移のシミュレーションに成功}}した~[\ref{Terashi_QML_Qdata}]。

NISQコンピュータでは、ノイズの影響を抑えるためにユニタリー演算をより少ない量子ゲートを使って実装することが望ましい。超伝導量子コンピュータでは、特に制御NOTゲートのエラーレートが1\%程度と高く、主要なエラー源である。本研究グループは回路に現れる制御NOTゲートごとに量子状態を測定し、得られる計算基底の出現頻度から不要な制御NOTゲートを削減する最適化ツール\mybf{AQCEL}を開発した~[\ref{Terashi_AQCEL}]。AQCELを使い、\mybf{\ul{計算精度を落とすことなく場の理論のシミュレーション回路に現れる制御NOTゲートを75\%程度削減できることを実証}}した~[\ref{Terashi_AQCEL}]。

研究分担者(飯山)は、CERN LHCのもう一つの大型実験であるCMS実験において、数十ペタバイトにも及ぶ実験データを管理するソフトウェアの構築・実装を行うなど、科学計算やソフトウェア開発に関する経験を多く有している~[\ref{Iiyama_Dynamo}]。また、\mybf{機械学習技術、特にエラー訂正デコーダとして使われるFPGAなど、低レイテンシ環境でのアルゴリズム実行に関して第一線での研究実績がある}~[\ref{Iiyama_GNN}]。2020年から量子コンピューティング研究に携わり、素粒子物理学への応用を念頭においた\mybf{\ul{量子状態生成アルゴリズムの考案}}や、\mybf{\ul{超伝導型量子ビットに代表される非調和振動子の物理シミュレーションの開発}}などを行ってきた。

研究分担者(陳)は、ATLAS実験においてデータ取得システムの運転とアップグレードに従事した豊富な経験がある。25ナノ秒毎に発生する陽子衝突を数十万以上の検出器チャンネルで記録し、その信号を帯域を逼迫させることなく正しいタイミングで確実に後段に送るという非常にシビアな要件のシステムであるが、それを満たすための開発や運用を通じて、高周波信号の時間制御やデータ圧縮・多重化に関する高いノウハウを有している。量子コンピュータのハードウェア開発においては、特に\mybf{マイクロ波の時間領域のエンジアリングを主導}しており、\mybf{\ul{レイテンシ改善を目的とした開発全般に専門的な適性がある}}。また、一昨年より超伝導量子ビットの製作や関連する表面微細加工の研究も先行して行ってきた。\mybf{\ul{既にジョセフソン接合や超伝導トランズモンの製作と測定に成功}}しており(図~\ref{fig:status}(b))、\mybf{\ul{本研究課題のハードウェア開発に必要な技術水準に達している}}。また、量子力学基礎論の理論研究~[\ref{Chen_QM}]や超伝導量子ビットを用いた暗黒物質の探索手法の提案~[\ref{Chen_DM}]など、基礎物理と量子効果、量子計測技術などに対しても鋭い感覚を有している。\vspace{-2mm}\\


\noindent{\bf (2)研究環境}\\
本研究では、東京大学素粒子物理国際研究センターで運用する計算機システムを使って、量子多体系シミュレーションやアルゴリズムのソフトウェア開発を行う。量子ビット数とともに状態ベクトル空間は指数的に増大するため、量子系の古典計算機シミュレーションには大容量メモリが必要になる。本研究グループでは、すでに\mybf{\ul{テラバイトスケールのメモリと高性能GPGPUを搭載する計算機を複数台運用}}しており、\mybf{\ul{シミュレータとして格子ゲージ理論の量子シミュレーションや量子機械学習を行うことが可能}}である。本研究では、エラー訂正デコーダのための古典計算(機械学習)アルゴリズムの開発や、一般的な回路ノイズモデルに対するシミュレータでの性能検証が重要であり、そのためにこの計算機環境をさらに拡張する。

本研究では、ジョセフソン接合ベースの超小型サーキュレータ/アイソレータとマイクロ波の多重化技術の開発を目指すが、そのためには超伝導量子素子の微細加工とマイクロ波エンジニアリングの環境が必要である。まず、微細加工には\mybf{東京大学武田先端知スーパークリーンルーム}にある共同利用施設を活用する。製作した超伝導素子の性能評価は、\mybf{東京大学低温科学センター・極低温プラットフォーム}の共同利用希釈冷凍機・高周波機器を使って行う。図~\ref{fig:status}(b)に示すように、\mybf{\ul{本研究グループは昨年から両共同利用施設を本格的に運用しており、既に設備をフル活用している}}。

%end 応募者の研究遂行能力及び研究環境 ====================

%begin 研究業績リスト ====================
\vspace*{1zw}
\noindent\textbf{参考文献} \vspace{-2mm}
\begin{enumerate}
	\setcounter{enumi}{4}
	\renewcommand{\labelenumi}{[\arabic{enumi}]}
	\item \label{Saito_CHEP} ``Quantum annealing algorithms for track pattern recognition'',
		M.~Saito, \me\ {\it et al.}, {\it EPJ Web Conf.} {\bf 245}, 10006 (2020).\vspace{-3mm}
	\item \label{Terashi_QML_selection} ``Event Classification with Quantum Machine Learning in High-Energy Physics'', 
		\me\ {\it et al.}, {\it Comput. Softw. Big Sci.} {\bf 5}, 2 (2021).\vspace{-3mm}
	%\item \label{Terashi_QML_HEP} ``Quantum Machine Learning in High Energy Physics'', 
	%	\me\ {\it et al.}, {\it Mach. Learn.: Sci. Technol.} {\bf 2}, 011003 (2021).\vspace{-2mm}
	%\item \label{Terashi_QC_HEP} ``Quantum Computing Applications in Future Colliders'', 
	%	H.~M.~Gray and \me, {\it Front. Phys.} {\bf 10}, 864823 (2022).\vspace{-2mm}
	\item \label{Terashi_QML_Qdata} ``Quantum data learning for quantum simulations in high-energy physics'', 
	L.~Nagano, \me\ {\it et al.}, arXiv:2306.17214.\vspace{-3mm}
	\item \label{Terashi_AQCEL} ``Initial-State Dependent Optimization of Controlled Gate Operations with Quantum Computer'', 
		W.~Jang, \me, \iiyama\  {\it et al.}, {\it Quantum} {\bf 6}, 798 (2022).\vspace{-3mm}	
	\item \label{Iiyama_Dynamo} ``Dynamo: Handling Scientific Data Across Sites and Storage Media'',
		\iiyama\ {\it et al.},  {\it Comput. Softw. Big Sci.} {\bf 5}, 2 (2021).\vspace{-3mm}
	\item \label{Iiyama_GNN} ``Distance-Weighted Graph Neural Networks on FPGAs for Real-Time Particle Reconstruction in High Energy Physics'',
		\iiyama\ {\it et al.}, {\it Front. Big Data} {\bf 3}, 44 (2021).\vspace{-3mm}
	\item \label{Chen_QM} ``Testing Bell's inequality using charmonium decays'', 
		\chen, Y.~Nakaguchi, S.~Komamiya, {\it Prog. Theor. Exp. Phys.}, 063A01 (2013).\vspace{-3mm}
	\item \label{Chen_DM} ``Detection of hidden photon dark matter using the direct excitation of transmon qubits",
		\chen\ {\it et al.}, arXiv:2212.03884.
\end{enumerate}
%end 研究業績リスト ====================

\input{pieces/p02_abilities_01}

%#Split: 03_rights  
%#PieceName: p03_rights
\input{pieces/p03_rights_00}
\section{3 人権の保護及び法令等の遵守への対応}
%    <<最大 1ページ>>

% s09_rights
%begin 人権の保護及び法令等の遵守への対応 ====================
該当しない。
%end 人権の保護及び法令等の遵守への対応 ====================

\input{pieces/p03_rights_01}

%#Split: 04_final_year  
%#PieceName: p04_final_year
\input{pieces/p04_final_year_00}
\section{4 研究計画最終年度前年度応募を行う場合の記述事項}
%    <<最大 1ページ>>

%s04_prep_finalyear
% 2020-08-16: Taku: Adjusted the horizontal position of the tabular.
%begin 最終年度の研究課題 ====================
\newcommand{\最終年度研究種目名}{}
\newcommand{\最終年度研究課題番号}{}
\newcommand{\最終年度研究課題名}{}
\newcommand{\最終年度研究期間}{}
%end 最終年度の研究課題 ====================
\input{pieces/p04_final_year_01}

\noindent
\textbf{当初研究計画及び研究成果}\\
%begin 研究計画最終年度の応募の計画と成果 ====================
%end 研究計画最終年度の応募の計画と成果 ====================
\\

\noindent
\textbf{前年度応募する理由}\\
%begin 研究計画最終年度の応募の理由 ====================
%end 研究計画最終年度の応募の理由 ====================

\input{pieces/p04_final_year_02}

%#Split: 99_tail
\input{pieces/hook9} % pieces
\end{document}

